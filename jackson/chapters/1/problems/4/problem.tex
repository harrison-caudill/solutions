\begin{question}

  Each of three charged spheres of radius $a$,

  \begin{enumerate}[label=\alph*]

  \item one conducting,

  \item one having a uniform charge density within its volume, and

  \item one having a spherically symmetric charge density that varies
    radially as $r^n$ ($n > -3$),

  \end{enumerate}
  has a total charge $Q$. Use Gauss's theorem to obtain the electric
  fields both inside and outside each sphere. Sketch the behavior of
  the fields as a function of radius for the first two spheres, and
  for the third with $n = -2, +2$

\end{question}

All of these problems will use Gauss' law
\begin{equation}
  \oint_S \vec{E} \cdot d\vec{a} = \frac{q_{enc}}{\epsilon_0}
\end{equation}


\subsubsection*{a) Conducting Sphere with Radius $a$}

The field inside the sphere is 0 and the field outside the sphere is
easily computed using Gauss' law:

\begin{equation}
  \frac{Q}{\epsilon_0}
  = \oint_S \vec{E} \cdot d\vec{a}
  = E \,4 \pi a^2
  \Rightarrow \vec{E} = \left. \frac{Q}{4 \pi \epsilon_0 r^2} \hat{r}\, \right|_{r>a\ \blacksquare}
\end{equation}



\subsubsection*{b) Non-Conducting Sphere with Radius $a$}

The field outside the sphere is easily computed with Gauss' law just
as it was for part (a).  The field inside the sphere is also easily
computed, but since only the charge inside of the surface contributes
to the E field, we must first compute that charge.

\begin{equation}
  q
  = Q \frac{V_{enc}}{V_{total}}
  = Q \frac{^4/_3 \pi r^3}{^4/_3 \pi a^3}
  = Q \frac{r^3}{a^3}
\end{equation}

Substituting our new value of the enclosed charge into our previous
equation, we have

\begin{equation}
  \vec{E} = \left. Q \frac{r^3}{a^3}\frac{1}{4 \pi \epsilon_0 r^2} \hat{r}\, \right|_{r \leq a}
  = \left. Q \frac{r}{a^3}\frac{1}{4 \pi \epsilon_0} \hat{r}\, \right|_{r \leq a\ \blacksquare}
\end{equation}


\subsubsection*{c) Non-Conducting Sphere with Radius $a$ and Non-Uniform $\rho$}

The field outside the sphere is easily computed with Gauss' law just
as it was for part (a).  Similarly to part (b), the field inside the
sphere can be easily computed by first computing the contained charge
as a function of radius:

\begin{equation}
  Q
  = \rho_0
  \int\limits_0^a
  \int\limits_0^\pi
  \int\limits_0^{2\pi}
  r^n
  r^2\,sin(\theta)
  \,dr
  \,d\theta
  \,d\varphi
  = \rho_0
  \int\limits_0^a
  \int\limits_0^\pi
  \int\limits_0^{2\pi}
  r^{n+2}
  sin(\theta)
  \,dr
  \,d\theta
  \,d\varphi
\end{equation}

Separating out our variables

\begin{equation}
  Q
  = \rho_0
  \int\limits_0^a
  r^{n+2}
  \,dr
  \int\limits_0^\pi
  sin(\theta)
  \,d\theta
  \int\limits_0^{2\pi}
  \,d\varphi
  = \frac{a^{n+3}}{n+3} \cdot 2 \cdot 2\pi
  = \frac{4\pi}{n+3} a^{n+3}
\end{equation}

From this we find an expression for our constant $\rho_0$

\begin{equation}
  \rho_0 = Q \frac{n+3}{4\pi a^{n+3}}
\end{equation}

Using our expression for $\rho_0$, we compute the enclosed charge as a
function of radius to be

\begin{equation}
  q
  = \rho_0 \frac{r^{n+3}}{n+3} \cdot 2 \cdot 2\pi
  = Q \frac{n+3}{4\pi a^{n+3}} \frac{r^{n+3}}{n+3} \cdot 4\pi
  = Q \frac{r^{n+3}}{a^{n+3}}
\end{equation}

Now that we have an expression for the contained charge, we sub that
into Gauss' law

\begin{equation}
  \vec{E}
  = \left. Q \frac{r}{a}^{n+3} \frac{1}{4 \pi \epsilon_0 r^2} \hat{r}\, \right|_{r \leq a\ \blacksquare}
\end{equation}
