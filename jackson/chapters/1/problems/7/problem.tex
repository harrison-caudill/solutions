
\begin{question}
Two long, cylindrical conductors of radii $a_1$ and $a_2$ are parallel
and separated by a distance $d$, which is large compared with either
radius. Show that the capacitance per unit length is given
approximately by

\begin{equation}
  C \approx \pi \epsilon_0 \left( ln \frac d a \right)^{-1}
\end{equation}


where $a$ is the geometrical mean of the two radii.

Approximately what gauge wire (state diameter in millimeters) would be
necessary to make a two-wire transmission line with a capacitance of
$1.2 * 10^{-11} F/m$ if the separation of the wires was $0.5 cm$? $1.5
cm$? $5.0 cm$?
\end{question}

As shown in figure \ref{fig::1::7::setup}, we will assume, contrary to
the text of the question, that the \emph{centers} of the condutors
will be separated by a distance $d$ rather than the conductors
themselves being separated by a distance $d$.

\begin{figure}
  \label{fig::1::7::setup}

  \begin{center}

    \includegraphics[width=\linewidth]{setup.png}

    \caption{The two conductors of radii $a_1$ and $a_2$ are separated by
      distance d at the center.  This geometry disagrees with the
      problem description, but agrees with the results.  Gaussian
      surfaces can be drawn around each conductor to determine E-Field
      strength arising from each conductor's charge.}

  \end{center}
\end{figure}
