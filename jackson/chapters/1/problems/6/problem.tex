\begin{question}

  A simple capacitor is a device formed by two insulated conductors
  adjacent to each other. If equal and opposite charges are placed on
  the conductors, there will be a certain difference of potential
  between them. The ratio of the magnitude of the charge on one
  conductor to the magnitude of the potential difference is called the
  capacitance (in SI units it is measured in Farads). Using Gauss's
  law, calculate the capacitance of

  \begin{enumerate}[label=\alph*]

  \item two large, flat, conducting sheets of area $A$, separated by a
    small distance $d$

  \item two concentric conducting spheres with radii $a$, $b$ where
    ($b > a$)

  \item two concentric conducting cylinders of length $L$, large
    compared to their radii $a$, $b$ where ($b > a$)

  \item What is the inner diameter of the outer conductor in an
    air-filled coaxial cable whose center conductor is a cylindrical
    wire of diameter 1mm and whose capacitance is $3 * 10^{11} F/m$?
    $3 * 10^{-12} F/m$?

  \end{enumerate}

\end{question}



\subsubsection*{a) Capacitance of Parallel Plates}
\begin{quote}
  Using Gauss's law, calculate the capacitance of two large, flat,
  conducting sheets of area $A$, separated by a small distance $d$
\end{quote}

\begin{figure}
  \begin{center}
    \includegraphics[width=2in]{a_setup}
    \caption{Two parallel plates of area $A$ and separated by a
      distance $d$ where $A^{1/2} \gg d$}
    \label{fig:1:6:a:setup}
  \end{center}
\end{figure}

In a parallel plate capacitor with a small separation relative to
sizes of the plates (as shown in figure \ref{fig:1:6:a:setup}) one can
ignore boundary effects.  Under the assumption that the E field
between the plates is uniform, we then employ Gauss' law (eq 1.9 in
the book) and as shown in figure \ref{fig:1:6:a:gauss}

\begin{equation}
  \oint_S \vec{E} \cdot d\vec{a} = \frac{q_{enc}}{\epsilon_0}
\end{equation}

\begin{figure}
  \begin{center}
    \includegraphics[height=2in]{a_gauss}
    \caption{Gaussian surfaces can be drawn around each plate to
      determine the magnitude of the E field between the plates.}
    \label{fig:1:6:a:gauss}
  \end{center}
\end{figure}

Since the field is uniform, we can infer that

\begin{equation}
  \oint_S \vec{E} \cdot d\vec{a}
  = 2EA
  = \frac{q_{enc}}{\epsilon_0}
  \Rightarrow E_1 = \frac{q}{2A\epsilon_0}
\end{equation}

Making note that the E field is from both plates, we see that

\begin{equation}
E = E_1 + E_2 = \frac{q}{A\epsilon_0}
\end{equation}

Going back to our objective

\begin{equation}
  C = \frac q V
\end{equation}

\begin{equation}
  V
  = \int{\vec{E} \cdot d\vec{l}}
  = d * E
  = d\frac{q}{A\epsilon_0}
\end{equation}

Substituting that into our definition of C gives us

\begin{equation}
  C
  = \frac q V
  = \frac{A\epsilon_0}{d}_\blacksquare
\end{equation}


\subsubsection*{b) Capacitance of Concentric Spheres}

\disputed{My solution disagrees with other external solutions in that
  I have a factor of 2 in there for the E field contribution from the
  outer shell's charge.  If you look in figure \ref{fig:1:6:b_wonky},
  you'll see 3 gaussian surfaces: A, B, and C.  Gaussian surface C
  tells us that the E-field outside of the outer shell is 0 because
  the total inner charge is 0.  Surface A tells us the E field
  contributed by the inner shell, but \emph{does not} tell us anything
  about the E field from any charges outside of it.  That brings us to
  surface B.  This one is, topologically, a sphere that surrounds the
  +q charge on the outer shell.  By noting that the surface integral
  of this one is going to effectively be C - A (the negative on A's
  integral is because the direction of dA is opposite) shows us that
  we get 0 - A.  That means that we get the contribution from both the
  inner and the outer shells.  When you play that out, I get a factor
  of 2 that results in a $2\pi$ at the end, whereas all published
  solutions manuals I can find say it results in $4\pi$.  Everyone
  else's solution involves only the charges in the middle.  It's fine
  to use Gauss' law to determine the E field from that central sphere,
  but it neglects the E field from anythinge else...like the outer
  sphere.}

\begin{quote}
  Using Gauss's law, calculate the capacitance of two concentric
  conducting spheres with radii $a$, $b$ where ($b > a$)
\end{quote}

As in part (a), computing the capacitance starts by computing $V$
which involves solving for $E$.  To compute $E$, we first turn to a
series of Gaussian surfaces shown in figure \ref{fig:1:6:b_wonky}.

\begin{figure}[h]
  \begin{center}
    \includegraphics[width=\linewidth]{b_wonky}
    \caption{This figure contains two charged spheres of radius $a$
      and $b$ and three Gaussian surfaces (A, B, and C).  Gaussian
      surface A is a simple sphere between the inner and outer shells;
      this shell captures the E field from the inner sphere only.
      Gaussian surface C is a simple sphere surrounding the outer
      sphere; it captures the E field from both surfaces combined.
      Gaussian surface B surrounds the outer shell, but also goes
      through an infinitesimal pinhole in the outer surface to then
      surround the inside of the outer shell.  This geometry allows
      this Gaussian surface to capture the E field contributions from
      the outer shell alone.}
    \label{fig:1:6:b_wonky}
  \end{center}
\end{figure}

\begin{equation}
1 = 2
\end{equation}

We can compute the E field contributed by the inner sphere quite
simply as

\begin{equation}
  \oint_S{\vec{E}\cdot d\vec{a}}
    = 4\pi r^2E
    = \frac{q}{\epsilon_0}
    \Rightarrow E_{inner} = \frac{q}{4\pi\epsilon_0}\frac{1}{r^2}
\end{equation}

To compute the E field contribution from the outer sphere, we first
differentiate the E field from the outer sphere inside the shell vs
outside the shell.  Applying Gauss' law to surface C in figure
\ref{fig:1:6:b_wonky} we find that:

\begin{equation}
  \oint_S{\vec{E}\cdot d\vec{a}} = \frac{q + -q}{\epsilon_0} = 0
    \Rightarrow \vec{E} = 0
\end{equation}

With that in mind, we can then examine Gaussian surface B which is a
combination of the inner and outer surfaces of the outer shell.  Since
we already know that the outer surface is 0, any E field from the
charge must be concentrated on the inside surface.

\begin{equation}
  \oint_S{\vec{E}\cdot d\vec{a}}
    = 4\pi r^2 E = \frac{q}{\epsilon_0}
    \Rightarrow E_{outer} = \frac{q}{4 \pi \epsilon_0} \frac{1}{r^2}
\end{equation}

Just as in the previous problem, we can easily show that the E fields
reenforce one another making their magnitudes additive:

\begin{equation}
  E(r) = E_{outer} + E_{inner}
  = \frac{q}{2 \pi \epsilon_0} \frac{1}{r^2}
\end{equation}

We then integrate the E field between the spheres to determine the
Voltage potential

\begin{equation}
  \begin{array}{lcl}
    V & = & \int \vec{E} \cdot d\vec{l}\\
    & = & \frac{q}{2 \pi \epsilon_0} \int_{a}^{b} \frac{1}{r^2}\\
    & = & - \frac{q}{2 \pi \epsilon_0} \left( \frac 1 b - \frac 1 a \right)\\
    & = & \frac{q}{2 \pi \epsilon_0} \left( \frac 1 a - \frac 1 b \right) \\ 
    & = & \frac{q}{2 \pi \epsilon_0} \left( \frac{b - a}{ab} \right) \\
  \end{array}
\end{equation}

Substituting V into our equation for capacitance we get

\begin{equation}
  C = \frac q V
  = \frac{2 \pi \epsilon_0}{\frac{b - a}{ab}}
  = 2 \pi \epsilon_0 \frac{ab}{b - a}
\end{equation}

\subsubsection*{c) Capacitance of Concentric Cylinders}

\disputed{This one is disputed for the same reason as (b)}

\begin{quote}
  Using Gauss's law, calculate the capacitance of two concentric
  conducting cylinders of length $L$, large compared to their radii
  $a$, $b$ where ($b > a$)
\end{quote}

Yet again, turning our eyes to Gauss' law:

\begin{equation}
  \oint_S \vec{E} \cdot d\vec{a} = \frac{q_{enc}}{\epsilon_0}
\end{equation}

In this case, we can draw Gaussian surfaces around the inner and outer
portions to determine the E field contributions.  Imagining the
cross-section shown in figure \ref{fig:1:6:b_wonky} to be for a
cylinder rather than for a sphere, and we can compute the E fields
just as we did in part (b) only for a cylinder.  The field at surface
A being computed as follows:

\begin{equation}
  \begin{array}{lcl}
    \oint_S \vec{E} \cdot d\vec{a}
    &=&\frac{q_{enc}}{\epsilon_0} \\
    &=& 2 \pi r L E(r) \\
    \Rightarrow E_{A}(r) &=& \frac{q}{2 \pi L \epsilon_0} \frac 1 r
  \end{array}
\end{equation}

As in part (b), the field at surface C is 0, meaning that the E field
contribution from the outer conductor is entirely on the interior
portion of surface B resulting in

\begin{equation}
  \begin{array}{lcl}
    \oint_S \vec{E} \cdot d\vec{a}
    &=&\frac{q_{enc}}{\epsilon_0} \\
    &=& 2 \pi r L E(r)\\
    \Rightarrow E_{B}(r) &=& -\frac{-q}{2 \pi L \epsilon_0} \frac 1 r\\
     &=& \frac{q}{2 \pi L \epsilon_0} \frac 1 r\\
     &=& E_{A}(r)
  \end{array}
\end{equation}

\begin{equation}
  E = E_A + E_B = 2E_A = \frac{q}{\pi L \epsilon_0} \frac 1 r
\end{equation}

Integrating to find the voltage, we see that

\begin{equation}
  V = \int_a^b{\vec{E}\cdot d\vec{l}}
    = \frac{q}{\pi L \epsilon_0} \int_a^b{\frac 1 r}
    = \frac{q}{\pi L \epsilon_0} (ln(b)-ln(a))
    = \frac{q}{\pi L \epsilon_0} ln \left(\frac b a\right)
\end{equation}

Solving then for C, we find that

\begin{equation}
C = \frac q v = \pi L \epsilon_0 \left(ln \frac b a \right)^{-1}
\end{equation}

\subsubsection*{d) Conductor Size}

\disputed{This one is disputed for the same reason as (b)}

Plugging in the numbers (with the 2 in there) you get a few mm vs a
hundred kilometers...it's pretty sensitive to the desired capacitance
assuming air as the dielectric material.
