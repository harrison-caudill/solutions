\begin{question}
  Using Dirac delta functions in the appropriate coordinates, express
  the following charge distributions as three-dimensional charge
  densities p(x).

  \begin{enumerate}[label=\alph*]

    \item In spherical coordinates, a charge $Q$ uniformly distributed
      over a spherical surface of radius $R$.

    \item In cylindrical coordinates, a charge $\lambda$ per unit
      length uniformly distributed over a cylindrical surface of
      radius $b$.

    \item In cylindrical coordinates, a charge $Q$ spread uniformly
      over a flat circular disc of negligible thickness and radius
      $R$.

    \item The same as part (c), but using spherical coordinates.
  \end{enumerate}

\end{question}


This problem is essentially looking for the appropriate $\rho$ so that
$\iiint_V \rho d\vec{V} = q$ in the chosen coordinate system.  In each
case, we're working with a uniform charge distribution so our first
step will be to ensure that the integral of some constant (usually
$\rho_0$, $\lambda_0$, or $\sigma_0$) multiplied by the physical
description of the charge (using delta and step functions) works out
as we'd expect.


\subsubsection*{a) Spherical Shell in Spherical Coordinates}

With a spherical shell, we're working with a surface charge density:

\begin{equation}
  \sigma = \frac{Q}{4\pi{}R^2}
\end{equation}

Physically describing a spherical shell in spherical coordinates yields:

\begin{equation}
  \delta(r-R)
\end{equation}

We now run a test integral to make sure our physical description of
the shell works out to the area of a shell.

\begin{equation}
  \int\limits_0^\infty \int\limits_0^\pi \int\limits_0^{2\pi}
  \delta(r-R) r^2sin(\theta)\, dr\, d\theta\, d\varphi
\end{equation}

Separating out our separable terms, we see that

\begin{equation}
  \int\limits_0^\infty r^2\delta(r-R) dr
  \int\limits_0^\pi sin(\theta) d\theta
  \int\limits_0^{2\pi} d\varphi
  = R^2 \cdot 2 \cdot 2\pi
  = 4\pi{}R^2
\end{equation}

Combining our charge density with our delta function we get

\begin{equation}
  \rho(r, \theta, \varphi) = \sigma\delta(r-R) = \frac{Q}{4\pi{}R^2}\delta(r-R)
  \ _\blacksquare
\end{equation}

\subsubsection*{b) Cylinderical Shell in Cylindrical Coordinates}

Note that we're using $r$ instead of $\rho$ so as not to have two
$\rho's$ in the same equation.  This problem works with a linear
charge density $\lambda$.  Converting that into a surface charge
density $\sigma$ we get

\begin{equation}
  \sigma = \frac{\lambda}{2 \pi{} b}
\end{equation}

Describing the charge shape, we propose that

\begin{equation}
  \rho = \sigma \cdot \delta(r - b)
\end{equation}

Taking our integral to ensure it works out, we have

\begin{equation}
  \int\limits_0^\infty
  \int\limits_0^{2\pi}
  \int\limits_0^L
  \sigma \delta(r-b)
  r
  \,dr
  \,d\varphi
  \,dz
\end{equation}

Once again separating out the components

\begin{equation}
  Q =
  \sigma
  \int\limits_0^\infty \delta(r-b)r\,dr
  \int\limits_0^{2\pi} d\varphi
  \int\limits_0^L dz
  = \sigma \cdot b \cdot 2\pi \cdot L
\end{equation}

Substituting in the value of sigma, we get

\begin{equation}
  Q = \frac{\lambda}{2 \pi{} b} \cdot b \cdot 2\pi \cdot L = \lambda L
  \ _\blacksquare
\end{equation}


\subsubsection*{c) Disc in Cylindrical Coordinates}

Same note as above about $r$ vs $\rho$, and once again, we're working
with a surface charge density $\sigma$

\begin{equation}
  \sigma = \frac{Q}{\pi R^2}
\end{equation}

We then guess the obvious configuration.  Note that the step function
is $U(R-r)$ as opposed to the more typical $U(r-R)$.  We do this so
that it's high and drops low at $R$ instead of starting low and going
high.

\begin{equation}
  \rho(r) = \delta(z) U(R-r)
\end{equation}

Taking our integral to check our results

\begin{equation}
  Q =
  \int\limits_0^\infty
  \int\limits_0^{2\pi}
  \int\limits_0^\infty
  \sigma \delta(z) U(R-r)
  r
  \,dr
  \,d\varphi
  \,dz
\end{equation}

Separating out our variables, and using the step function to restrict
the upper bound of the integration of the radius to $R$

\begin{equation}
  Q =
  \sigma
  \int\limits_0^R
  r
  \,dr
  \int\limits_0^{2\pi}
  \,d\varphi
  \int\limits_0^\infty
  \delta(z)
  \,dz
  = \sigma \cdot \frac{R^2}{2} \cdot 2\pi \cdot 1
  = \sigma \pi R^2
  \ _\blacksquare
\end{equation}


\subsubsection*{d) Disc in Spherical Coordinates}

This is the only one where we find a variable dependence.  As with the
others, we start with a charge density $\sigma$

\begin{equation}
  \sigma = \frac{Q}{\pi R^2}
\end{equation}

Using our charge density we posit a distribution function based upon
the geometry of the problem (once again using $R-r$ instead of $r-R$)

\begin{equation}
  \rho = \sigma \delta(\theta - \frac \pi 2) U(R - r)
\end{equation}

Setting up our integral, and crossing our fingers for a $Q$ at the end

\begin{equation}
  Q =
  \int\limits_0^\infty
  \int\limits_0^\pi
  \int\limits_0^{2\pi}
  \sigma \delta(\theta - \frac \pi 2) U(R - r)
  r^2sin(\theta)
  \, dr
  \, d\theta
  \, d\varphi
\end{equation}

Separating out our variables and using the step function to limit the
bounds of integration

\begin{equation}
  Q =
  \sigma
  \int\limits_0^R
  r^2
  \, dr
  \int\limits_0^\pi
  sin(\theta)
  \delta(\theta - \frac \pi 2)
  \, d\theta
  \int\limits_0^{2\pi}
  \, d\varphi
  = \sigma \cdot \frac{R^3}{3} \cdot 1 \cdot 2\pi
  \neq \sigma \pi R^2
\end{equation}

Clearly, something isn't right.  There are a few ways to approach this one:

\begin{enumerate}
  \item We can look more closely at how the delta function interacts
    with the d$\theta$.

  \item We can use the known appropriate result $\sigma 4 \pi R^2$ to
    find the correct distribution.

  \item We can guess the solution (most people do it this way)

  \item Some crazy reasoning that I'm not sure I understand/agree with
    that you'll find in one of the other solution manuals out there.
\end{enumerate}

Starting with point 1, our integral from earlier can be rewritten as 

\begin{equation}
  Q =
  \sigma
  \int\limits_0^R
  \int\limits_0^\pi
  \int\limits_0^{2\pi}
  \delta(\theta - \frac \pi 2)
  \,r sin(\theta) \, d\varphi
  \,r \, d\theta
  \, dr
\end{equation}

Recalling the volume element for spherical integration

\begin{equation}
  dV = r\,sin(\theta)\,d\varphi r\,d\theta\ dr
\end{equation}

Conceptually, there's a factor of $r$ in there which, when constrained
with a $\delta$ function leaves an excess $r$ in the result.  Adding
$\frac 1 r$ eliminates the problem (as we'll show soon).

The next easy way (point 2) to do this is to start with the known
surface area of a disc and work backwards -- you'll find the same
result.

Plugging in the new factor

\begin{equation}
  \rho = \sigma \frac{\delta(\theta - \frac \pi 2)}{r} U(R - r)
\end{equation}

Setting up our integral, and crossing our fingers for a $Q$ at the end

\begin{equation}
  Q =
  \int\limits_0^\infty
  \int\limits_0^\pi
  \int\limits_0^{2\pi}
  \sigma \frac{\delta(\theta - \frac \pi 2)}{r} U(R - r)
  r^2sin(\theta)
  \, dr
  \, d\theta
  \, d\varphi
\end{equation}

Separating out our variables and using the step function to limit the
bounds of integration

\begin{equation}
  Q =
  \sigma
  \int\limits_0^R
  r
  \, dr
  \int\limits_0^\pi
  sin(\theta)
  \delta(\theta - \frac \pi 2)
  \, d\theta
  \int\limits_0^{2\pi}
  \, d\varphi
  = \sigma \cdot \frac{R^2}{2} \cdot 1 \cdot 2\pi
  = \sigma \pi R^2
  \ _\blacksquare
\end{equation}
