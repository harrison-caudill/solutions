\begin{question}
Use Gauss's theorem [and (1.21) if necessary] to prove the following:

  \begin{enumerate}[label=\alph*]

    \item Any excess charge placed on a conductor must lie entirely on
      its surface. (A conductor by definition contains charges capable
      of moving freely under the action of applied electric fields.)

    \item A closed, hollow conductor shields its interior from fields
      due to charges outside, but does not shield its exterior from
      the fields due to charges placed inside it.

    \item The electric field at the surface of a conductor is normal
      to the surface and has a magnitude $\sigma / \epsilon_0$, where
      $\sigma$ is the charge density per unit area on the surface.

  \end{enumerate}

\end{question}

These problems seem trivial at first glance, but really aren't.  To
understand why, let us first examine Gauss' law:
\begin{equation}
  \oint_S \vec{E} \cdot d\vec{a} = \frac{q_{enc}}{\epsilon_0}
\end{equation}

While Gauss' law does provide insight into E fields from charges
within the surface, it does not provide any insight into E field
magnitudes from charges outside of the surface.

\subsubsection*{a) Charges on the Surface}

\begin{quote}
Any excess charge placed on a conductor must lie entirely on its
surface. (A conductor by definition contains charges capable of moving
freely under the action of applied electric fields.)
\end{quote}

Starting with equation 1.1

\begin{equation}
\vec{F} = q\vec{E}
\end{equation}

If a Gaussian surface were to be drawn anywhere entirely within the
conductor (e.g. \emph{just} inside) then it would imply a non-zero
E-field contained within the conductor.  Since a conductor, by
definition, permits charges to move freely in response to the presence
of an E field, we know that any correct arrangement will involve
stationary charges in equilibrium.

With that stipulation, there are, then, exctly two scencarios (and one
superposition) to consider: all of the charges concentrated on the
interior surface and the charges concentrated on the exterior surface.

If we draw a Gaussian surface \emph{just} inside the conductor so as
to just encompass the inner surface, then any charge would result in a
net E field right at the inner surface.  Since all of the excess
charges would, by definition, be the same they wold repel each other.
This mutual repulsion will force the charges to move into the middle.

Since the middle of the conductor has already been shown to be
unavailable, that leaves only the external surface.  A Gaussian
surface drawn just under the exterior surface would have no charge
within, and thus the E field would be implying no forces on the
charges.

\subsubsection*{b) Hollow Conductor Charge Shielding}

\begin{quote}
A closed, hollow conductor shields its interior from fields due to
charges outside, but does not shield its exterior from the fields due
to charges placed inside it.
\end{quote}

While Gauss' law very specifically \emph{does not} provide information
about external E fields, it \emph{does} allow us to show (as in part a
above) that the charges lie entirely on the surface.  Because the
charges would be under acceleration in the presence of a net force,
and we stipulate that the system is in equilibrium, we know that the
net force on the charges is 0, therefor we know that the local E field
around the charges on the surface is 0.  With a local field of
approximately 0, and no enclosed charge, we know that the interiror
field will be 0.

The second part is easy to show.  Any Gaussian surface drawn inside
the shell and around the enclosed charge will be non-zero and thus the
shell would fail to shield against any enclosed charge.

\subsubsection*{c) Local E Field is Perpendicular to the Surface}

\begin{quote}
The electric field at the surface of a conductor is normal to the
surface and has a magnitude $\sigma / \epsilon_0$, where $\sigma$ is
the charge density per unit area on the surface.
\end{quote}

We start by drawing a small loop intersecting the surface of a charged
conductor, as showin in figure \ref{fig:1:1:loopy}

\begin{figure}
  \includegraphics[width=4in]{perpendicular.png}
  \caption{A closed loop intersecting the exterior surface of a
    charged conductor.}
\label{fig:1:1:loopy}
\end{figure}

Using Equation 1.21

\begin{equation}
  \oint_S \vec{E} \cdot d\vec{l} = 0
\end{equation}

we can see that the loop integral in figure \ref{fig:1:1:loopy} must
evaluate to 0.  Since the E field inside the conductor is 0, we know
that the integral along the segment C is 0.  Since the loop can be
drawn arbitrarily small, we can use local linearity to show that B and
D are equal in magnitude and opposite in sign.  For the loop integral
as a whole to evaluate to 0, then, the A segment must also be 0.
Since the A and C segments are necessarily 0, then the E field must be
perpendicular to the surface.

To show the magnitude of the E field, we once again use Gauss' law.  A
Gaussian surface with upper surface area $A$.  In that case

\begin{equation}
  \oint_S{\vec{E} \cdot d\vec{A} = q_{enc} / \epsilon_0}
  \Rightarrow E = \frac{q_{enc}}{A} \frac{1}{\epsilon_0}
\end{equation}

Since $q = \sigma A$ we can rewrite it as

\begin{equation}
  E = \frac{\sigma}{\epsilon_0} _\blacksquare
\end{equation}
