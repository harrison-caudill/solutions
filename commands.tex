%%%%%%%%%%%%%%%%%%%%%%%%%%%%%%%%%%%%%%%%%%%%%%%%%%%%%%%%%%%%%%%%%%%%%%%%%%%%%%%
%% Generally Useful Commands                                                 %%
%%%%%%%%%%%%%%%%%%%%%%%%%%%%%%%%%%%%%%%%%%%%%%%%%%%%%%%%%%%%%%%%%%%%%%%%%%%%%%%

\newcounter{rowcount}
\newcommand{\rowit}{
  \stepcounter{rowcount}
  \therowcount
}

\def\bul{$\bullet$}
\def\deg{$^{\circ}$}
\def\bul{$\bullet$}

\newcommand{\fixme}[1]{
  {\bf FIXME:}(#1)
}

\newcommand{\dmz}{
  \begin{minipage}{\linewidth}
    \vspace{0.2in}
    \hrule
    \hrule

    \par
    \vspace{0.05in}

    \begin{center}
      DMZ
    \end{center}

    \hrule
    \hrule
    \vspace{0.2in}
  \end{minipage}
}


%%%%%%%%%%%%%%%%%%%%%%%%%%%%%%%%%%%%%%%%%%%%%%%%%%%%%%%%%%%%%%%%%%%%%%%%%%%%%%%
%% Problems                                                                  %%
%%%%%%%%%%%%%%%%%%%%%%%%%%%%%%%%%%%%%%%%%%%%%%%%%%%%%%%%%%%%%%%%%%%%%%%%%%%%%%%

%% \definekeys{problem}{
%%   path .store in=\problem@path
%% %%  path .initial=3cm,
%% }
%\newcoommand{\problem}{
%% \NewDocumentCommand{\problem}{m}{%
%% \def\path{#1}
%% %  \setkeys{\problem}{#1}%
%% %  \problem@path
%% \path
%% }

\newenvironment{question}{%
\begin{quote}
}{%
\end{quote}
}

\NewDocumentCommand{\probChapter}{m}{
\setcounter{section}{#1-1}
\section{Chapter #1}
\problem{1}{7}

}

\NewDocumentCommand{\problem}{mm}{%
\setcounter{subsection}{#2-1}
\subsection{Problem #2}
\graphicspath{{\buildPath/\bookName/chapters/#1/problems/#2}}

\begin{question}
Two long, cylindrical conductors of radii $a_1$ and $a_2$ are parallel
and separated by a distance $d$, which is large compared with either
radius. Show that the capacitance per unit length is given
approximately by

\begin{equation}
  \frac C L \approx \pi \epsilon_0 \left( ln \frac d a \right)^{-1}
\end{equation}


where $a$ is the geometrical mean of the two radii.

Approximately what gauge wire (state diameter in millimeters) would be
necessary to make a two-wire transmission line with a capacitance of
$1.2 * 10^{-11} F/m$ if the separation of the wires was $0.5 cm$? $1.5
cm$? $5.0 cm$?
\end{question}

As shown in figure \ref{fig:1:7:setup}, we will assume, contrary to
the text of the question, that the \emph{centers} of the condutors
will be separated by a distance $d$ rather than the conductors
themselves being separated by a distance $d$.

\begin{figure}[h]
  \begin{center}

    \includegraphics[width=\linewidth]{setup.png}

    \caption{The two conductors of radii $a_1$ and $a_2$ are separated by
      distance d at the center.  This geometry disagrees with the
      problem description, but agrees with the results.  Gaussian
      surfaces can be drawn around each conductor to determine E-Field
      strength arising from each conductor's charge.}

    \label{fig:1:7:setup}

  \end{center}
\end{figure}

Starting with the characteristic equation for a capacitor
\begin{equation}
  C = \frac q V
\end{equation}

then with the definition of $V$

\begin{equation}
    V \equiv \int_P \vec{E} \cdot d\vec{l}
\end{equation}

We know from the physics of the problem that the Electric field
effects from both conductors are additive.  Youc an provie this to
yourself by drawing two Gaussian surfaces and being mindful of the
signs of both the charge and the field direction.  Once you are
sufficiently convinced that the effects are additive, we can begin the
integral to find $V_1$ and $V_2$ (with $V$ of course being equal to
$V_1 + V_2$).

\begin{figure}[h]
  \begin{center}

    \includegraphics[width=\linewidth]{surfaces.png}

    \caption{Gaussian surfaces can be drawn along the length of the
      conductors to find the E-field contribution from each conductor.
      Each integral is valid from the boundary of one condutor to the
      boundary of the other.}

    \label{fig:1:7:gauss}

  \end{center}
\end{figure}


Starting with Gauss' law (eq 1.9 in the book)

\begin{equation}
  \oint_S \vec{E} \cdot d\vec{a} = \frac{q_{enc}}{\epsilon_0}
\end{equation}

\begin{equation}
  \label{eq:1:7:e}
  2 \pi r L E = \frac{q_{enc}}{\epsilon_0}
  \Rightarrow E = \frac{q_{enc}}{2 \pi L \epsilon_0} \frac 1 r
\end{equation}

\begin{equation}
  \label{eq:1:7:v1}
  V_1 = \int_{a_1}^{d-a_2}{\vec{E} \cdot d\vec{l}}
  = \frac{q}{2 \pi L \epsilon_0} \int_{a_1}^{d-a_2}{\frac{dr}{r}}
  = \frac{q}{2 \pi L \epsilon_0} \left( ln(d-a_2) - ln(a_1)\right)
  = \frac{q}{2 \pi L \epsilon_0} ln\left( \frac{d-a_2}{a_1} \right)
\end{equation}

Subsituting in the variables for $V_2$, we get

\begin{equation}
  \label{eq:1:7:v2}
  V_2 = \int_{a_2}^{d-a_1}{\vec{E} \cdot d\vec{l}}
  = \frac{q}{2 \pi L \epsilon_0} \int_{a_2}^{d-a_1}{\frac{dr}{r}}
  = \frac{q}{2 \pi L \epsilon_0} \left( ln(d-a_1) - ln(a_2y)\right)
  = \frac{q}{2 \pi L \epsilon_0} ln\left( \frac{d-a_1}{a_2} \right)
\end{equation}

Solving for $V$, we find that

\begin{equation}
  V = V_1 + V_2
  = \frac{q}{2 \pi L \epsilon_0} \left[ ln\left( \frac{d-a_2}{a_1} \right) + ln\left( \frac{d-a_1}{a_2} \right) \right]
  = \frac{q}{2 \pi L \epsilon_0} ln\left( \frac{(d-a_2)(d-a_1)}{a_1a_2} \right)
\end{equation}

Substituting in the geometric mean of $a_1$ and $a_2$ we get

\begin{equation}
  a = (a_1a_2)^{(1/2)}
  \Rightarrow V = \frac{q}{2 \pi L \epsilon_0} ln\left( \frac{(d-a_2)(d-a_1)}{a^2} \right)
\end{equation}

Since the problem specifies that $d \gg a_1$ and $d \gg a_2$, we can
simplify the expression for $V$ to

\begin{equation}
  V \approx \frac{q}{2 \pi L \epsilon_0} ln\left( \frac{(d)(d)}{a^2} \right)
\end{equation}

Moving the $\frac 1 2$ into the ln, we see that

\begin{equation}
  V \approx \frac{q}{\pi L \epsilon_0} ln\left( \frac{d}{a} \right)
\end{equation}

Subsituting our expression for $V$ into the definition of $C$ and we get

\begin{equation}
  C = \frac q V
  \approx \frac{q}{\frac{q}{\pi L \epsilon_0} ln\left( \frac d a \right)}
  = \pi L \epsilon_0 ln\left( \frac d a \right)^{-1}
\end{equation}

Pulling out the factor of L, we find that

\begin{equation}
  \frac C L
  \approx \pi \epsilon_0 \left( ln \frac d a \right)^{-1}_\blacksquare
\end{equation}





}


%%%%%%%%%%%%%%%%%%%%%%%%%%%%%%%%%%%%%%%%%%%%%%%%%%%%%%%%%%%%%%%%%%%%%%%%%%%%%%%
%% Generally Useful Environments                                             %%
%%%%%%%%%%%%%%%%%%%%%%%%%%%%%%%%%%%%%%%%%%%%%%%%%%%%%%%%%%%%%%%%%%%%%%%%%%%%%%%

\newenvironment{packedEnum}{%
  \begin{enumerate}
    \setlength{\itemsep}{0pt}
    \setlength{\parskip}{0pt}
    \setlength{\parsep}{0pt}
}{%
  \end{enumerate}
}

\newenvironment{packedItem}{%
  \begin{large}
  \begin{itemize}
    \setlength{\itemsep}{5pt}
    \setlength{\parskip}{0pt}
    \setlength{\parsep}{0pt}
}{%
  \end{itemize}
  \end{large}
}


%%%%%%%%%%%%%%%%%%%%%%%%%%%%%%%%%%%%%%%%%%%%%%%%%%%%%%%%%%%%%%%%%%%%%%%%%%%%%%%
%% Section & Heading Management                                              %%
%%%%%%%%%%%%%%%%%%%%%%%%%%%%%%%%%%%%%%%%%%%%%%%%%%%%%%%%%%%%%%%%%%%%%%%%%%%%%%%

\newcommand{\jobSectionTitle}[1]{%
  \noindent
  \textbf{\large{#1}}
}%

\newcommand{\jobSection}[5]{%
  \def\jobTitle{#1}
  \def\jobOrg{#2}
  \def\jobStart{#3}
  \def\jobEnd{#4}
  \def\jobPath{#5}

  \ifthenelse{\equal{\jobStart}{}}{
    %% We don't have any dates (just the job)
    \jobSectionTitle{\jobTitle{}, \jobOrg{}}
  }{
    \ifthenelse{\equal{\jobEnd}{}}{
      %% There isn't a specified job end date, skip it
      \jobSectionTitle{\jobTitle{}, \jobOrg{} (\jobStart)}
    }{
      %% Use the job end date specified
      \jobSectionTitle{\jobTitle{}, \jobOrg{} (\jobStart{} \bul{} \jobEnd{})}
    }
  }

  \ifthenelse{\equal{\jobPath}{}}{%
  }{%
    \input{\jobPath}
  }%
  \vspace{8pt}
}

\newcommand*\ruleline[1]{%
  \vspace{0.1in}
  \par
  \noindent
  \raisebox{.8ex}{%
    \makebox[\linewidth]{%
      \hrulefill\hspace{1ex}\raisebox{-.8ex}{#1}\hspace{1ex}\hrulefill
    }%
  }%
}%

\newcommand{\sectionHeader}[1]{%
  \ruleline{{\bf \Large #1}}
}


%%%%%%%%%%%%%%%%%%%%%%%%%%%%%%%%%%%%%%%%%%%%%%%%%%%%%%%%%%%%%%%%%%%%%%%%%%%%%%%
%% Cover Letters (General)                                                   %%
%%%%%%%%%%%%%%%%%%%%%%%%%%%%%%%%%%%%%%%%%%%%%%%%%%%%%%%%%%%%%%%%%%%%%%%%%%%%%%%

\newenvironment{jobDescriptor}{%
  %% FIXME: Figure this one out properly
  \vspace{-10pt}

  \begin{itemize}
    \setlength{\itemsep}{3pt}
    \setlength{\parskip}{0pt}
    \setlength{\parsep}{0pt}
}{%
  \end{itemize}%
  \vspace*{-\baselineskip}
}

\newenvironment{coverLetter}{%
  \begin{large}
}{%
  \end{large}%
}

\newenvironment{jobStory}{%
  \hfill\begin{minipage}{\dimexpr\textwidth-\parindent}
  \begin{flushleft}
  %%\begin{adjustwidth}{1.5em}{}
}{%
  \end{flushleft}
  \end{minipage}
  %%\end{adjustwidth}
}

\newenvironment{positionList}{%
  \begin{large}
  \begin{itemize}
}{%
  \end{itemize}
  \end{large}
}


%%%%%%%%%%%%%%%%%%%%%%%%%%%%%%%%%%%%%%%%%%%%%%%%%%%%%%%%%%%%%%%%%%%%%%%%%%%%%%%
%% Cover Letters (Qualifications Table)                                      %%
%%%%%%%%%%%%%%%%%%%%%%%%%%%%%%%%%%%%%%%%%%%%%%%%%%%%%%%%%%%%%%%%%%%%%%%%%%%%%%%

%% To use kwargs in LaTeX is a bit complicated -- we really need a
%% pythonic wrapper for LaTeX...my calling?
%%
%% 1) create a make-at-letter block to define the keys and default values
%%
%%     \makeatletter
%%     \define@key{qualTableArgs}{tocTitle}{\def\tocTitle{#1}}
%%     \define@key{qualTableArgs}{includeSummary}{\def\includeSummary{#1}}
%%     \setkeys{qualTableArgs}{includeSummary=true,tocTitle=}
%%     \makeatother
%%
%%
%% 2) Define your command/environment properly where
%%    n = number of positional arguments
%%
%%    \newcommand{\contribution}[n+1][]{%
%%
%%
%% 3) Snarf the arguments (both positional and keyword):
%%
%%    \setkeys{qualTableArgs}{#1} <-- Grabbing kwargs
%%    \def\secTitle{#2}           <-- Grabbing first positional argument
%%    \def\authorName{#3}         <-- Grabbing second positional argument
%%    ...
%%    \def\fname{#n+1}            <-- Grabbing the nth positional argument
%%
%% 4) Use the values (you know, the thing you wanted to do from the
%%    start).  This is actually the easy part, you just use them like
%%    you would any other LaTeX command:
%%
%%    kwargs[''tocTitle''] = \tocTitle
%%
%%
%% 5) Pass in some kwargs
%%
%%    \whatever[myKwarg=false]{positional}{arguments}{as}{usual}

\makeatletter
\define@key{qualTableArgs}{notes}{\def\notes{#1}}
\setkeys{qualTableArgs}{notes=}
\makeatother

\newenvironment{qualificationsTables}{%
  \begin{tabular}{cl}
    \Circle{} & Requirement unmet.\\
    \LEFTcircle{} & Requirement partially met.\\
    \CIRCLE{} & Requirement fully met.\\
  \end{tabular}
}{%
}

\newenvironment{qualificationsTable}[2][]{%

  \setkeys{qualTableArgs}{#1}
  \def\title{#2}

  \newpage
  \thispagestyle{plain}
  
  %% Table wide options
  \renewcommand{\arraystretch}{1.8}
  \rowcolors{2}{white}{gray!25}
  %%\tabcolsep=0pt

  {\centering}

  %%\begin{center}
    \begin{longtable}{cm{0.43\linewidth}m{0.43\linewidth}}

      %% First Page Header
      \rowcolor{lightgray}\multicolumn{3}{c}{%
        \ifthenelse{\equal{\notes}{}}{%
          {\bfseries \Large \title}%
        }{%
          {\bfseries \Large \title}%

          \notes
        }%
      }%
      \\

      \rowcolor{lightgray}
      & \multicolumn{1}{c}{\textbf{Job Requirement}}
      & \multicolumn{1}{c}{\textbf{Experience}} \\
      \hline 
      \endfirsthead

    % Subsequent Headers
      \hline
      \rowcolor{lightgray}
      & \multicolumn{1}{c}{\textbf{Job Requirement}}
      & \multicolumn{1}{c}{\textbf{Experience}} \\
      \hline 
      \endhead

      % Page Footers
      \hline
      \multicolumn{3}{r}{{Continued on next page}} \\
      \hline
      \endfoot

      % Final Footer
      \hline
      \hline
      \endlastfoot
}{%
    \end{longtable}
  %%\end{center}
}

\makeatletter
\define@key{qualItemArgs}{fullyMet}{\def\fullyMet{#1}}
\define@key{qualItemArgs}{partiallyMet}{\def\partiallyMet{#1}}
\define@key{qualItemArgs}{notMet}{\def\notMet{#1}}
\setkeys{qualItemArgs}{fullyMet=true, partiallyMet=false, notMet=false}
\setkeys{qualItemArgs}{fullyMet=true}
\makeatother%

\newcommand{\qualTableOptionals}{%
  \hline
  \multicolumn{3}{c}{%
    {\bfseries \large Secondary Requirements}%
  }\\
  \hline
}

\newcommand{\qualTableItem}[3][]{%

  \setkeys{qualItemArgs}{#1}

  \ifthenelse{\boolean{\notMet}}{%
    \Circle%
    & {\normalsize #2}%
    & {\normalsize #3}%
    \\
  }{%
    \ifthenelse{\boolean{\partiallyMet}}{%
      \LEFTcircle%
      & {\normalsize #2}%
      & {\normalsize #3}%
      \\
    }{%
      \CIRCLE
      & {\normalsize #2}%
      & {\normalsize #3}%
      \\
    }%
  }%
}

\newcommand{\coverIntro}[1]{%

  \begin{large}

    \ifthenelse{\equal{#1}{}}{%
      \input{covers/default}
    }{%
      #1
    }

    \begin{tabular}{cl}
      \Circle{} & Requirement unmet.\\
      \LEFTcircle{} & Requirement partially met.\\
      \CIRCLE{} & Requirement fully met.\\
    \end{tabular}

  \end{large}
}
