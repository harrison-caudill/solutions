\documentclass{article}

%%%%%%%%%%%%%%%%%%%%%%%%%%%%%%%%%%%%%%%%%%%%%%%%%%%%%%%%%%%%%%%%%%%%%%%%%%%%%%%
%%                             Math Stuff                                    %%
%%%%%%%%%%%%%%%%%%%%%%%%%%%%%%%%%%%%%%%%%%%%%%%%%%%%%%%%%%%%%%%%%%%%%%%%%%%%%%%
\usepackage{amsmath}        % math utility
\usepackage{amssymb}        % math symbols
\usepackage{esint}          % loop integrals


%%%%%%%%%%%%%%%%%%%%%%%%%%%%%%%%%%%%%%%%%%%%%%%%%%%%%%%%%%%%%%%%%%%%%%%%%%%%%%%
%%                         General Utility                                   %%
%%%%%%%%%%%%%%%%%%%%%%%%%%%%%%%%%%%%%%%%%%%%%%%%%%%%%%%%%%%%%%%%%%%%%%%%%%%%%%%
\usepackage{geometry}       % Page layout
\usepackage{calc}           % Perform math operations on counters
\usepackage{xfrac}          % For some reason need this for \graphicspath


\geometry{
  letterpaper,
  top=.75in,
  bottom=.75in,
  left=.75in,
  right=.75in,
  paperwidth=7in,
  paperheight=10in
}


\begin{document}

\input{BUILD/bookParams}
\graphicspath{{\buildPath/\bookName}}


\title{Quick Reference}
\author{Harrison Caudill}
\date{}
\maketitle

\section{Jackson's Emag}

\begin{figure}
  \begin{center}
  \includegraphics[width=5in]{coordinates_diagram}
  \caption{Spherical coordinates on the left, cylindrical on the
    right.}
  \end{center}
  \label{fig:coordinate-diagrams}
\end{figure}

\subsection{Spherical Integration}

See figure \ref{fig:coordinate-diagrams} for a vector diagram.

\begin{itemize}

\item \textbf{$Azimuth (\varphi)$:} We use $0 - 2\pi$ instead of
  $-\pi - \pi$

\item \textbf{$Elevation (\theta)$:} Jackson uses $0 - \pi$ which
  means that it's from $+z$ instead of being from the $xy$ plane.
  Gotta remember to use $sin(\theta)$ instead of $cos(\theta)$.

\end{itemize}

\begin{equation}
  \int\limits_{r=0}^R \int\limits_{\theta=0}^{\pi} \int\limits_{\varphi=0}^{2\pi}
  f(r, \theta, \varphi)\ r^2sin(\theta)\ dr\ d\varphi\ d\theta
\end{equation}


\subsection{Cylindrical Integration}

See figure \ref{fig:coordinate-diagrams} for a vector diagram.

\begin{itemize}

\item \textbf{$Azimuth (\varphi)$:} We use $0 - 2\pi$ instead of
  $-\pi - \pi$

\item \textbf{$Radial Distance (\rho)$:} Typical $r$ in polar
  coordinates.

\end{itemize}

\begin{equation}
  \int\limits_{\rho=0}^R \int\limits_{\varphi=0}^{2\pi} \int\limits_{z=0}^{Z}
  f(\rho, \varphi, z)\ \rho\ d\rho\ d\varphi\ dz
\end{equation}

\end{document}
